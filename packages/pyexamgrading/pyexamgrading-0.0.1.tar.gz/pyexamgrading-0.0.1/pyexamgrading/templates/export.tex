\documentclass[10pt]{article}
\usepackage[a4paper]{geometry}
\usepackage{amsfonts}
\usepackage{booktabs}
\usepackage{siunitx}
\geometry{a4paper,margin=15mm}

\begin{document}
\pagenumbering{gobble}
\renewcommand{\arraystretch}{1.25}

<%def name="decimal_value(value)">\
%if value.denominator == 1:
${value.numerator}\
%else:
${f"{float(value):.1f}"}\
%endif
</%def>

<%def name="show_fraction(value)">\
%if value.denominator == 1:
${value.numerator}\
%else:
\frac{${value.numerator}}{${value.denominator}}\
%endif
</%def>

<%def name="pretty_fraction(value, show_zero = True, decimal_percent = False)">\
%if value.denominator == 1:
%if show_zero or value.numerator != 0:
%if not decimal_percent:
${value.numerator}\
%else:
${value.numerator * 100}\%\
%endif:
%else:
\
%endif
%else:
<%
	if decimal_percent:
		text_value = f"{float(value) * 100:.1f}"
		is_exact = (fractions.Fraction(text_value) == 100 * value)
	else:
		text_value = f"{float(value):.1f}"
		is_exact = (fractions.Fraction(text_value) == value)
	equals = "=" if is_exact else "\\approx"
%>\
$\frac{${value.numerator}}{${value.denominator}} ${equals} ${text_value}$${"\\%" if decimal_percent else ""}\
%endif
</%def>

<%def name="render_entry(entry)">
\newpage
\begin{center}
\begin{tabular}{ll} 
	\toprule
	{\textbf{Prüfungsfach}} & {${exam.name}}\\%
	{\textbf{Datum}} & {${exam.date}}\\%
	{\textbf{Prüfender}} & {${exam.lecturer}}\\%
	{\textbf{Prüfling}} & {${entry.student.full_name} (${entry.student.course})}\\%
	{\textbf{Matrikelnummer}} & {${entry.student.student_number}}\\%
	\bottomrule
\end{tabular}
\end{center}

\vspace{0.5cm}
\subsection*{Prüfungsteilleistungen}
<%
	cumulative = fractions.Fraction(0)
%>
\begin{tabular}{llllr} 
	\toprule
	\textbf{Name} & \textbf{Ergebnis} & \textbf{Faktor} & \textbf{Gewichtet} & \textbf{Kumulativ}\\%
	\midrule
	%for (name, contribution) in entry.grade.breakdown_by_task.items():
	<%
		cumulative += contribution.scaled_points
	%>\
	{${name}} &\
	{${decimal_value(contribution.original_points)} / ${decimal_value(contribution.task.max_points)}} &\
	{${pretty_fraction(contribution.task.scalar)}} &\
	{${pretty_fraction(contribution.scaled_points, show_zero = False)}} &\
	{${f'{cumulative:.3f}'}}\
	\\%
	%endfor
	\bottomrule
\end{tabular}

\vspace{0.5cm}
\subsection*{Ergebnisermittlung}
<%
	ratio = entry.grade.grade.achieved_points / entry.grade.grade.max_points
	percent = ratio * 100
%>
\begin{tabular}{ll} 
	\toprule
	{\textbf{Erreichte Punkte}} & {${pretty_fraction(entry.grade.grade.achieved_points)} von ${pretty_fraction(entry.grade.grade.max_points)}}\\%
	{\textbf{Verhältnis}} & {${pretty_fraction(ratio, decimal_percent = True)}}\\%
	{\textbf{Notenschema}} & \
%if exam.grading_scheme.grading_scheme_type == GradingSchemeType.GermanUniversityLinear:
Universität linear ${f'{exam.grading_scheme.parameters['cutoff_low'] * 100:.1f}'}\% bis ${f'{exam.grading_scheme.parameters['cutoff_high'] * 100:.1f}'}\%\\%
%elif exam.grading_scheme.grading_scheme_type == GradingSchemeType.GermanUniversityCutoff:
Universität cutoff ${f'{exam.grading_scheme.parameters['cutoff_low'] * 100:.1f}'}\% bis ${f'{exam.grading_scheme.parameters['cutoff_high'] * 100:.1f}'}\%\\%
%else:
${error}
%endif
	{\textbf{Berechnung}} & \
%if exam.grading_scheme.grading_scheme_type in [ GradingSchemeType.GermanUniversityLinear, GradingSchemeType.GermanUniversityCutoff ]:
$1 + 3 \cdot \frac{${show_fraction(exam.grading_scheme.parameters['cutoff_high'])} - ${show_fraction(ratio)}}{${show_fraction(exam.grading_scheme.parameters['cutoff_high'])} - ${show_fraction(exam.grading_scheme.parameters['cutoff_low'])}} \rightarrow ${entry.grade.grade.text}$\
%else:
${error}
%endif
	\\%
	{\textbf{Nächst bessere Note}} &\
%if entry.grade.next_best_grade is None:
	--\
%else:
	${f"{entry.grade.next_best_grade.point_difference:.1f}"} Punkt${"" if (entry.grade.next_best_grade.point_difference == 1) else "e"} fehlen zur Note ${entry.grade.next_best_grade.grade.text}\
%endif
	\\%
	\bottomrule
\end{tabular}

%if len(entries) >= min_participants_stats:
\vspace{0.5cm}
\subsection*{Statistische Informationen}
\begin{tabular}{ll} 
	\toprule
	{\textbf{Gesamtzahl Arbeiten}} & {${len(entries)}}\\%
	{\textbf{Notendurchschnitt}} & {${f"{stats.average_grade:.1f}"}}\\%
	{\textbf{Von Ihnen erreichter Perzentil}} & {${f"{stats.percentile[entry.grade.grade.text]:.0f}"}}\%\\%
	\bottomrule
\end{tabular}
%endif

</%def>

%for entry in entries:
${render_entry(entry)}
%endfor

\end{document}
